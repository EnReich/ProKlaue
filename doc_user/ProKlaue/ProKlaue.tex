\documentclass[a4paper, openany, oneside]{memoir}
\usepackage[T1]{fontenc}
\usepackage{lmodern}
\usepackage[utf8]{inputenc}
\usepackage[german]{babel}
\usepackage{graphicx}
\usepackage{listings}
\usepackage{color}
\usepackage{siunitx}
\usepackage[inline]{enumitem}
\usepackage[obeyspaces, hyphens]{url}
\usepackage{hyperref}

\graphicspath{{./img/}}


\pretitle{\begin{center}\Huge\bfseries}
\title{User Guide für das ProKlaue Plugin}
\posttitle{\par\vskip1em{\normalfont\normalsize Installation unter Windows\par\vfill}\end{center}}
\author{Kai Hainke}
\predate{\vfill\begin{center}\large}
\chapterstyle{thatcher}

\DeclareUrlCommand{\dir_path}{\def\UrlLeft{\textrm{»}}\def\UrlRight{\textrm{«}}}

\DeclareUrlCommand{\File_path}{\def\UrlLeft{\textrm{}}\def\UrlRight{\textrm{}}}

\DeclareUrlCommand{\file_ending}{\def\UrlLeft{\textrm{}}\def\UrlRight{\textrm{}}}


\definecolor{dkgreen}{rgb}{0,0.6,0}
\definecolor{gray}{rgb}{0.5,0.5,0.5}
\definecolor{mauve}{rgb}{0.58,0,0.82}
\lstset{frame=tb,
  language=Python,
  aboveskip=3mm,
  belowskip=3mm,
  showstringspaces=false,
  columns=flexible,
  basicstyle={\small\ttfamily},
  numbers=none,
  numberstyle=\tiny\color{gray},
  keywordstyle=\color{blue},
  commentstyle=\color{dkgreen},
  stringstyle=\color{mauve},
  breaklines=true,
  tabsize=3
}



\begin{document}



\maketitle


%\pagebreak
%\dirpath{Documents\BliBla\bludsdus\dssdsd}\linebreak\linebreak
%\url{Documents\BliBla\bludsdus\dssdsd}
%\pagebreak

\chapter{Installation von python Packages für den Maya Python Interpreter}
Grundsätzlich gibt es zwei Varianten, die benötigten python libs zu installieren. Die erste Möglichkeit ist ein möglichst exakt gleichen Python-Interpreter (Version, Compiler-Version, Bit-Architektur, etc.) und die Packages dafür zu installieren und dann zu kopieren. Die zweite, hier empfohlene, Variante für Windows ist die Packages \href{https://drive.google.com/drive/folders/0BwsYd1k8t0lEfjJqV21yTnd2elVhNXEyTXhHclhxbDhvWVF5WWZUeVFISWViaFh1TzhrNTQ}{hier} herunterzuladen, zu entpacken und in den \dir_path{C:\Program Files\Autodesk\Maya2014\Python\lib\site-packages} Ordner zu verschieben. Die zur Zeit benötigten packages sind:
\begin{itemize}
\item numpy
\item scipy
\item scikit-learn (sklearn) (momentan nur benutzt bei den Achsen und Druck Scripts)
\end{itemize}
 
  

\chapter{Installieren des Plugins}
Zur Installation des Plugins lädt man sich die aktuelle Version des Plugins aus dem \href{https://github.com/EnReich/ProKlaue}{Git} herunter (Clone or download \(\rightarrow\) Download ZIP). Entpacken und Verschieben in den Ordner \dir_path{Autodesk\maya<version>\bin\plug-ins}, so dass \File_path{proKlaue.py} direkt in diesem Verzeichnis liegt. 

In Maya unter Windows \(\rightarrow\) Settings/Preferences \(\rightarrow\) Plugin Manager \(\rightarrow\) Refresh. Nach dem Eintrag \File_path{proKlaue.py} suchen und \texttt{Loaded} und \texttt{Auto load} auswählen. Zuletzt Windows \(\rightarrow\) Settings/Preferences \(\rightarrow\) Preferences \(\rightarrow\) Setting \(\rightarrow\) Selection \(\rightarrow\) \texttt{Track selection order} aktivieren. 

Für mehr Informationen ist die Dokumentation zu konsultieren. Zu finden unter \File_path{ProKlaue/doc/_build/html/index.html}.


\chapter{Informationen zum Projekt}
\section{Struktur}
Einen groben Überblick gibt die folgende Liste:
\begin{itemize}
\item \dir_path{scripts} - Ordner mit Skripten (R/Python):
\begin{itemize}
\item \File_path{angles.py} - Script zum Tracken der Winkel in einer animierten Szene mit JCS (nach Groot und Suntay)
\item \File_path{druck.R} - R Skript für die Zusammenführung von Fußungsfläche und Druckdaten
\item \File_path{pressureStatistics.py} - Python Skript für die Zusammenführung von Fußungsfläche und Druckdaten (mit \File_path{~UtilFunctions.py} für Funktionsdefinitionen)
\item \File_path{calculateJointCS.py} - Python Skript für die Berechnung von Joint Coordinate Systems
\item \File_path{axesToAnimated.py} - Python Skript für die Überführung von berechneten JCS von einer Szene in eine andere (bspw. von neutral zu animiert)
\item \File_path{plotHeatmaps.R.py} - R Skript zum Plotten von Heatmaps mit Hilfe vorher erstellter Daten aus dem Maya-Command cmds.altitudeMap
\end{itemize}
\item \dir_path{pk_src} - Ordner mit Maya Commands des Plugins, insbesondere
\begin{itemize}
\item \File_path{overlapStatistics.py} zum Erstellen von Statistiken zur Überdeckung von zwei oder mehreren Objekten (relativ robuste Berechnung von Schnittvolumen), siehe auch \File_path{intersection.py} und \File_path{vhacd.py}
\item \File_path{altitudeMap.py} zum Erstellen von Höhenmaps
\item \File_path{frontVertices.py} zum Abspeichern aller Vorderseiten-Segmente/Punkte von einer gegebenen Plane aus 
\item \File_path{normalize.py} zum Ausrichten von Objekten anhand der Eigenvektoren ihrer Kovarianz-Matrix
\item \File_path{misc.py} mit verschiedenen, nützlichen, all-purpose Funktionen
\end{itemize}
\item \dir_path{bin} - Executables für die VHACD Berechnung
\item \dir_path{doc} - Sphynx Dokumentation
\item \dir_path{doc_user} - Guides, Latex+pdf-Dateien
\item \dir_path{testdaten} - Testdateien und mitunter auch Plots
\end{itemize}
\pagebreak
\section{Nützliche Dev-Tools}
Für die Entwicklung der Scripts/Plugins bieten sich einige Tools an, lediglich eine Empfehlung, aber vlt. hilft es ja:
\begin{itemize}
\item RStudio für R
\item PyCharm für Python
\item MayaCharm (PyCharm Plugin zur Interaktion mit Maya)
\item Sublime für das Anzeigen großer Textdateien
\item Sphynx für Dokumentation
\item git zur Versionierung
\end{itemize}


\end{document}